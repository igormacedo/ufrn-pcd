% !TEX encoding = UTF-8 Unicode
% !TEX root = ../../relatorio.tex

%% Responsavel:

\subsection{Questão 3.11}

Finding \textbf{prefix sums} is a generalization of global sum. Rather than simply finding the sum of $n$ values,
\begin{equation*}
  x_{0} + x_{1} + ... + x_{n-1},
\end{equation*}
the prefix sums are the $n$ partial sums
\begin{equation*}
  x_{0}, x_{0} + x_{1} , x_{0} + x_{1} + x_{2} , ... , x_{0} + x_{1} + ... + x_{n-1}.
\end{equation*}


a. Devise a serial algorithm for computing the $n$ prefix sums of an array with n elements.

b. Parallelize your serial algorithm for a system with $n$ processes, each of which is storing one of the $x_{i}$s.

c. Suppose $n = 2^{k}$ for some positive integer $k$. Can you devise a serial algorithm and a parallelization of the serial algorithm so that the parallel algorithm requires only $k$ communication phases?

d. MPI provides a collective communication function, \texttt{MPI\_Scan} , that can be used to compute prefix sums:

\begin{lstlisting}[language=C]
int MPI_Scan(
      void*           sendbuf_p   /* in  */,
      void*           recvbuf_p   /* out */,
      int             count       /* in  */,
      MPI_Datatype    datatype    /* in  */,
      MPI_Op          op          /* in  */,
      MPI_Comm        comm        /* in  */);
\end{lstlisting}

It operates on arrays with \texttt{count} elements; both \texttt{sendbuf\_p} and \texttt{recvbuf\_p} should refer to blocks of \texttt{count} elements of type \texttt{datatype}. The \texttt{op} argument is the same as \texttt{op} for \texttt{MPI\_Reduce}. Write an MPI program that generates a random array of count elements on each MPI process, finds the prefix sums, and prints the results.

a.
\lstinputlisting[language=C]{sections/q3.11/code-a.c}

b.
\lstinputlisting[language=C]{sections/q3.11/code-b.c}

c. \textbf{Serial}
\lstinputlisting[language=C]{sections/q3.11/code-c1.c}
\textbf{Paralelo}
\lstinputlisting[language=C]{sections/q3.11/code-c2.c}

d. \lstinputlisting[language=C]{sections/q3.11/code-d.c}

Output:  [vec] => [sum] \\
rank 0 - [3 6 7 5 3 ] => [3 6 7 5 3 ] \\
rank 1 - [0 9 8 5 1 ] => [3 15 15 10 4 ] \\
rank 2 - [6 5 8 0 5 ] => [9 20 23 10 9 ] \\
rank 3 - [1 3 4 6 3 ] => [10 23 27 16 12 ] \\
rank 4 - [5 5 0 2 6 ] => [15 28 27 18 18 ] \\

%%% Local Variables:
%%% mode: latex
%%% TeX-master: "../../relatorio"
%%% End:
