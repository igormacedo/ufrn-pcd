% !TEX encoding = UTF-8 Unicode
% !TEX root = ../../relatorio.tex

%% Responsavel:

\subsection{Questão 3.5}

In a binary tree, there is a unique shortest path from each node to the root. The length of this path is often called the depth of the node. A binary tree in which every nonleaf has two children is called a full binary tree, and a full binary tree in which every leaf has the same depth is sometimes called a complete binary tree. See Figure 3.14. Use the principle of mathematical induction to prove that if T is a complete binary tree with n leaves, then the depth of the leaves is log 2 (n).\\

Para fazer a indução vamos considerar $\mathrm{log}_2(n) = d$, onde d é a profundidade.

Então,
\begin{equation}
  \mathrm{log}_2(n) = d \implies 2^{d} = n
\end{equation}

Considerando o caso base em que d = 0,
\begin{equation} \label{eq1}
\begin{split}
2^d & = n \\
2^0 & = n \\
2^0 & = 1
\end{split}
\end{equation}

Tomamos que $d = k$ e assumimos que $2^k = n$ é verdadeiro. Então, aplicamos o passo de indução, onde $d = k + 1$, e para a próxima profundidade, $n_{i} = n_{i - 1}\cdot2$. Logo,
\begin{equation}
\begin{split}
2^{k+1} & = n\cdot 2 \\
        & = 2^{k} 2^{1} \\
        & = 2^{k+1}
\end{split}
\end{equation}

Portanto, $2^{d} = n$ e, logo $log_{2}(n) = d$, onde n é o número de folhas e d é a profundidade.

%%% Local Variables:
%%% mode: latex
%%% TeX-master: "../../relatorio"
%%% End:
