% !TEX encoding = UTF-8 Unicode
% !TEX root = ../../relatorio.tex

%% Responsavel:
\subsection{Questão 3.1}

What happens in the greetings program if, instead of \texttt{strlen(greeting) + 1}, we use \texttt{strlen(greeting)} for the length of the message being sent by processes \texttt{1, 2, ..., comm\_sz- 1} ? What happens if we use \texttt{MAX\_STRING} instead of \texttt{strlen(greeting) + 1} ? Can you explain these results?\\

Neste caso, o \texttt{+ 1} indica que o caractere de terminação da string também deve ser incluido no envio da mensagem. Se substituirmos por apenas \texttt{strlen(greeting)} a mensagem pode ser impressa corretamente ou não, dependendo do conteúdo presente no buffer de recebimento. Caso o buffer de recebimento esteja preenchido com zeros ("\textbackslash0"), o comando \texttt{printf()} vai conseguir imprimir a mensagem corretamente mesmo que não exista um terminador nulo na mensagem enviada.

Em testes feitos localmente, as mensagens sempre foram exibidas corretamente, pois os buffers estavam sempre sendo iniciados com zero em suas posições de memória.

%%% Local Variables:
%%% mode: latex
%%% TeX-master: "../../relatorio"
%%% End:
