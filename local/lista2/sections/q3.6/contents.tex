% !TEX encoding = UTF-8 Unicode
% !TEX root = ../../relatorio.tex

%% Responsavel:

\subsection{Questão 3.6}

Suppose comm\_sz = 4 and suppose that x is a vector with n = 14 components.

a. How would the components of x be distributed among the processes in a program that used a block distribution?

b. How would the components of x be distributed among the processes in a program that used a cyclic distribution?

c. How would the components of x be distributed among the processes in a program that used a block-cyclic distribution with blocksize b = 2?

You should try to make your distributions general so that they could be used regardless of what comm sz and n are. You should also try to make your distributions “fair” so that if q and r are any two processes, the difference between the number of components assigned to q and the number of components assigned to r is as small as possible.

\begin{table}[h!]
\centering
\begin{tabular}{||c c c c||}
 \hline
 Processos & Bloco & Cíclico & Bloco-Cíclico \\
 & & &  (tamanho do bloco = 2)\\ [0.5ex]
 \hline\hline
 0 & 0, 1, 2, 3 & 0, 4, 8, 12 & 0 1, 8 9 \\
 1 & 4, 5, 6, 7 & 1, 5, 9, 13 & 2 3, 10 11\\
 2 & 8, 9, 10, & 2, 6, 10, & 4 5, 12 13 \\
 3 & 11, 12, 13, & 3, 7, 11, & 6 7 \\
 \hline
\end{tabular}
\caption{Distribuição dos elementos do vetor}
\label{table:1}
\end{table}

%%% Local Variables:
%%% mode: latex
%%% TeX-master: "../../relatorio"
%%% End:
