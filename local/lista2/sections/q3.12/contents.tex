% !TEX encoding = UTF-8 Unicode
% !TEX root = ../../relatorio.tex

%% Responsavel:

\subsection{Questão 3.12 *}

An alternative to a butterfly-structured allreduce is a ring-pass structure. In a ring-pass, if there are p processes, each process q sends data to process q + 1, except that process p - 1 sends data to process 0. This is repeated until each process has the desired result. Thus, we can implement allreduce with the following code:

\begin{lstlisting}[language=C]
sum = temp val = my val;
  for (i = 1; i < p; i++) {
    MPI_Sendrecv_replace(&temp_val, 1, MPI_INT, dest,
    sendtag, source, recvtag, comm, &status);
    sum += temp val;
}
\end{lstlisting}


a. Write an MPI program that implements this algorithm for allreduce. How does its performance compare to the butterfly-structured allreduce?

b. Modify the MPI program you wrote in the first part so that it implements prefix sums.
%%% Local Variables:
%%% mode: latex
%%% TeX-master: "../../relatorio"
%%% End:
