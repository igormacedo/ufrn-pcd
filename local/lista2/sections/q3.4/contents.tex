% !TEX encoding = UTF-8 Unicode
% !TEX root = ../../relatorio.tex

%% Responsavel:

\subsection{Questão 3.4}

Modify the program that just prints a line of output from each process
( \texttt{mpi\_output.c} ) so that the output is printed in process rank order: process 0s output first, then process 1s, and so on.\\

\lstinputlisting[language=C, firstline=1, lastline=27]{sections/q3.4/code.c}

O princípio da resulução desta questão é perceber que para lidar com o não-determinismo do output de um programa MPI, nós devemos mandar todas as mensagens de saída para um único processo, neste caso o processo 0. Dessa forma, apenas um processo é responsável por gerenciar as mensagens de saída e, assim, podemos controlar a ordem de saída das mensagens.
%%% Local Variables:
%%% mode: latex
%%% TeX-master: "../../relatorio"
%%% End:
