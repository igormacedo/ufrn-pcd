% !TEX encoding = UTF-8 Unicode
\documentclass[paper=a4, fontsize=11pt, oneside]{scrartcl}
\setcounter{secnumdepth}{5}
\setcounter{tocdepth}{5}
\usepackage[brazil]{babel}
\usepackage{listings}
\usepackage[T1]{fontenc}
\usepackage[utf8]{inputenc}
%\usepackage{minted}
\usepackage[table,xcdraw]{xcolor}
\usepackage{ulem}
\usepackage{afterpage}
\usepackage{amsmath}
\usepackage{array}
\usepackage{authblk}
\usepackage{datetime}
\usepackage{enumerate}
\usepackage{ifthen}
\usepackage{pdflscape}
\usepackage[absolute]{textpos}
\usepackage{times}
\usepackage{url}
\usepackage{hyperref}
\usepackage{float}
\usepackage{booktabs}
 \usepackage{booktabs}
 \usepackage[table,xcdraw]{xcolor}
 %\documentclass[xcolor=table]{beamer}
\usepackage[top=3cm, bottom=2cm, left=3cm, right=2cm]{geometry}
\usepackage[portuguese, ruled, linesnumbered]{algorithm2e}
\usepackage{graphicx}
\usepackage{tabu}
\usepackage{subcaption}
\DeclareGraphicsExtensions{.pdf,.png,.jpg}
\DeclareUnicodeCharacter{00A0}{~}

\lstdefinelanguage{JavaScript}{
  keywords={break, case, catch, continue, debugger, default, delete, do, else, finally, for, function, if, in, instanceof, new, return, switch, this, throw, try, typeof, var, void, while, with},
  morecomment=[l]{//},
  morecomment=[s]{/*}{*/},
  morestring=[b]',
  morestring=[b]",
  sensitive=true
}


\usepackage{tocloft}
\renewcommand{\cfttoctitlefont}{\Large\bfseries}
\renewcommand{\cftsecfont}{\bfseries}
\renewcommand{\cfttoctitlefont}{\LARGE\bfseries}
\renewcommand{\cftloftitlefont}{\LARGE\bfseries}
\renewcommand{\cftlottitlefont}{\LARGE\bfseries}
\renewcommand{\cftdot}{.}

\renewcommand{\contentsname}{Conteúdo}

\usepackage{titlesec}
\titleformat*{\section}{\Large\bfseries}
\titleformat*{\subsection}{\large\bfseries}

\usepackage{natbib}
\setlength{\bibsep}{5pt}

\usepackage{fancyhdr}
\pagestyle{fancyplain}
\fancyhead[L]{}
\fancyhead[C]{}
\fancyhead[R]{\footnotesize\thepage}
\fancyfoot{}
\renewcommand{\headrulewidth}{0pt}
\renewcommand{\footrulewidth}{0pt}

\fancypagestyle{empty}{
\fancyhf{}
\fancyhead{}
\fancyfoot[L]{}
\fancyfoot[C]{}
\fancyfoot[R]{}
}
\fancypagestyle{lscape}{
\fancyhf{}
\fancyhead[L]{}
\fancyhead[C]{}
\fancyhead[R]{
\begin{textblock}{1}(0.75,2){\rotatebox{90}{\footnotesize \thepage}}
\end{textblock}
}
\fancyfoot{}
\renewcommand{\headrulewidth}{0pt}
\renewcommand{\footrulewidth}{0pt}
}

\linespread{1.15}

\titleformat{\paragraph}
{\normalfont\normalsize\bfseries}{\theparagraph}{1em}{}
\titlespacing*{\paragraph}
{0pt}{3.25ex plus 1ex minus .2ex}{1.5ex plus .2ex}

\providecommand{\printwp}{}
\newcommand{\wpname}[1]
{
   \renewcommand{\printwp}{
      \ifthenelse{\equal{#1}{WP2}}{WP2 -- Aplicações}{}
      \ifthenelse{\equal{#1}{WP3}}{WP3 -- Sensoriamento}{}
      \ifthenelse{\equal{#1}{WP4}}{WP4 -- Infraestrutura}{}
      \ifthenelse{\equal{#1}{WP5}}{WP5 -- Middleware}{}
      \ifthenelse{\equal{#1}{WP6}}{WP6 -- Análise e Visualização de Dados}{}
   }
}

\definecolor{mygreen}{rgb}{0,0.6,0}
\definecolor{mygray}{rgb}{0.5,0.5,0.5}
\definecolor{mymauve}{rgb}{0.58,0,0.82}

\lstset{language=Matlab,%
    %basicstyle=\color{red},
    breaklines=true,%
    morekeywords={matlab2tikz},
    keywordstyle=\color{blue},%
    morekeywords=[2]{1}, keywordstyle=[2]{\color{black}},
    identifierstyle=\color{black},%
    stringstyle=\color{mymauve},
    commentstyle=\color{mygreen},%
    showstringspaces=false,%without this there will be a symbol in the places where there is a space
    numbers=left,%
    numberstyle={\tiny \color{black}},% size of the numbers
    numbersep=9pt, % this defines how far the numbers are from the text
    emph=[1]{for,end,break},emphstyle=[1]\color{red}, %some words to emphasise
    %emph=[2]{word1,word2}, emphstyle=[2]{style},
}

\usepackage{xcolor}

\colorlet{punct}{red!60!black}
\definecolor{background}{HTML}{EEEEEE}
\definecolor{delim}{RGB}{20,105,176}
\colorlet{numb}{magenta!60!black}

\lstdefinelanguage{json}{
    basicstyle=\normalfont\ttfamily,
    numbers=left,
    numberstyle=\scriptsize,
    stepnumber=1,
    numbersep=8pt,
    showstringspaces=false,
    breaklines=true,
    frame=lines,
    backgroundcolor=\color{white},
    literate=
     *{0}{{{\color{numb}0}}}{1}
      {1}{{{\color{numb}1}}}{1}
      {2}{{{\color{numb}2}}}{1}
      {3}{{{\color{numb}3}}}{1}
      {4}{{{\color{numb}4}}}{1}
      {5}{{{\color{numb}5}}}{1}
      {6}{{{\color{numb}6}}}{1}
      {7}{{{\color{numb}7}}}{1}
      {8}{{{\color{numb}8}}}{1}
      {9}{{{\color{numb}9}}}{1}
      {:}{{{\color{punct}{:}}}}{1}
      {,}{{{\color{punct}{,}}}}{1}
      {\{}{{{\color{delim}{\{}}}}{1}
      {\}}{{{\color{delim}{\}}}}}{1}
      {[}{{{\color{delim}{[}}}}{1}
      {]}{{{\color{delim}{]}}}}{1},
}

% !TEX encoding = UTF-8 Unicode
% !TEX root = ../relatorio.tex

\title{Segunda Lista de Exercícios -- 3.1-3.11,3.13-3.25}

\author{
Prof. Samuel Xavier - DCA/UFRN\\

\textbf{Aluno}\\
\\
Igor Macedo Silva - Bacharelando em Engenharia de Computação \\
}

\date{[Outubro de 2017]}

\makeatletter
\def\@maketitle{
\begin{center}
   \includegraphics[width=7.01cm]{cover/imgs/ufrn.png}\\
   \vfill
   {\large Universidade Federal do Rio Grande do Norte\\
   Departamento de Engenharia de Computacão e Automação}

   \vskip 4em
   {\Large Programação Concorrente e Distribuida}

   \vskip 4.5em
   {\normalsize\printwp}

   \vskip 5em
   {\LARGE\bfseries\@title}

   \vfill
   {\normalsize Natal-RN, Brasil\\\@date}
\end{center}

\newpage
{\noindent\Large\bfseries Professor}\smallskip\\
{\normalsize\@author}
}
\makeatother


\begin{document}

\frenchspacing
\maketitle
\thispagestyle{empty}

\newpage
\tableofcontents
\thispagestyle{empty}

\newpage
\listoffigures
\thispagestyle{empty}

\newpage
\listoftables
\thispagestyle{empty}

%%% Local Variables:
%%% mode: latex
%%% TeX-master: "../relatorio"
%%% End:

\newpage

% Introduction
\section{Descrição}

Lista da 2a unidade
Descrição:
Apresentar as respostas a todas as questões de exercício do livro texto, com exceção às questões: 3.3, 3.15 e 3.18.

Instruções para resolução (LEIAM!):
\begin{itemize}
  \item Procure responder corretamente todas as questões da lista;
  \item Suas respostas serão validadas de forma oral por amostragem - geralmente de 2 à 3 defesas orais;
  \item Se não conseguir responder alguma questão, procure esclarecer as dúvidas em tempo em sala de aula com o professor, pelo SIGAA, com um colega, ou por e-mail. Se necessário, é possível marcar um horário para tirar dúvidas na sala do professor;
  \item Não serão aceitas respostas "mágicas", ou seja, quando a resposta está na lista entregue mas você não sabe explicar como chegou a ela. Sua nota nesse caso será 0 (zero). Mesmo que não saiba explicar apenas parte da sua resposta;
  \item Procure entregar a resolução da lista de forma organizada. Isso pode favorecer a sua nota;
  \item Os códigos dos programas requisitados (ou as partes relevantes) deverão aparecer no corpo da resolução da questão;
  \item A resolução da lista deverá ser entregue em formato PDF em apenas 1 (um) arquivo;
  \item O envio da resolução pode ser feito inúmeras vezes. Utilize-se disso para manter sempre uma versão atualizada das suas respostas e evite problemas com o envio próximo ao prazo de submissão devido a instabilidades no SIGAA;
  \item A lista com o número das questões respondidas deve aparecer na primeira folha da lista. Não será aceita alteração nessa lista.
  \item Procure preparar sua defesa oral para cada questão. Explicações diretas e sem arrodeios favorecerão a sua nota;
  \item A defesa deverá ser agendada com antecedência. Para isso, indique por email (samuel@dca.ufrn.br) no mínimo 3 horários dentro dos intervalos disponíveis em pelo menos 3 turnos diferentes. Caso não tenha disponibilidade em 3 turnos diferentes, deverá apresentar uma justificativa.
  \item Os horários disponíveis serão disponibilizados em uma notícia na turma virtual e serão atualizados a medida que os agendamentos forem sendo fixados.
  \item A defesa oral leva apenas de 10 a 15 minutos em horários fixados com antecedência. Não será tolerado que o aluno chegue atrasado para a sua prova.

\end{itemize}

Período:
Inicia em 20/09/2017 às 00h00 e finaliza em 11/10/2017 às 23h59

\newpage

\section{Questões}
\label{sec:andamento}

% !TEX encoding = UTF-8 Unicode
% !TEX root = ../../relatorio.tex

%% Responsavel:
\subsection{Questão 5.8}

Consider the loop:
\begin{lstlisting}[language=C]
a[0] = 0;
for(i = 1; i < n; i++)
  a[i] = a[i - 1] + i ;
\end{lstlisting}

There’s clearly a loop-carried dependence, as the value of a[i] can’t be computed without the value of a[i - 1]. Can you see a way to eliminate this dependence and parallelize the loop?

Analisando o resultado do loop, vemos que o vetor vai armazenar a soma de uma sequência de números, uma série aritmética que pode ser descrita como:

\begin{equation}
  S_{n} = \frac{n}{2} \cdot (a_{1} + a_{n})
\end{equation}

Onde $a_{1}$ é o termo inicial, $a_{n}$ é o termo final e $n$ é o número de termos a serem somados. Logo, podemos traduzir essa formula em um loop for que irá calcular cada elemento do vetor:

\begin{lstlisting}[language=C]
for(i = 0; i < n; i++)
  a[i] = i*(i+1)/2 ;
\end{lstlisting}



%%% Local Variables:
%%% mode: latex
%%% TeX-master: "../../relatorio"
%%% End:

% !TEX encoding = UTF-8 Unicode
% !TEX root = ../../relatorio.tex

%% Responsavel:
\subsection{Questão 5.8}

Consider the loop:
\begin{lstlisting}[language=C]
a[0] = 0;
for(i = 1; i < n; i++)
  a[i] = a[i - 1] + i ;
\end{lstlisting}

There’s clearly a loop-carried dependence, as the value of a[i] can’t be computed without the value of a[i - 1]. Can you see a way to eliminate this dependence and parallelize the loop?

Analisando o resultado do loop, vemos que o vetor vai armazenar a soma de uma sequência de números, uma série aritmética que pode ser descrita como:

\begin{equation}
  S_{n} = \frac{n}{2} \cdot (a_{1} + a_{n})
\end{equation}

Onde $a_{1}$ é o termo inicial, $a_{n}$ é o termo final e $n$ é o número de termos a serem somados. Logo, podemos traduzir essa formula em um loop for que irá calcular cada elemento do vetor:

\begin{lstlisting}[language=C]
for(i = 0; i < n; i++)
  a[i] = i*(i+1)/2 ;
\end{lstlisting}



%%% Local Variables:
%%% mode: latex
%%% TeX-master: "../../relatorio"
%%% End:

% !TEX encoding = UTF-8 Unicode
% !TEX root = ../../relatorio.tex

%% Responsavel:
\subsection{Questão 5.8}

Consider the loop:
\begin{lstlisting}[language=C]
a[0] = 0;
for(i = 1; i < n; i++)
  a[i] = a[i - 1] + i ;
\end{lstlisting}

There’s clearly a loop-carried dependence, as the value of a[i] can’t be computed without the value of a[i - 1]. Can you see a way to eliminate this dependence and parallelize the loop?

Analisando o resultado do loop, vemos que o vetor vai armazenar a soma de uma sequência de números, uma série aritmética que pode ser descrita como:

\begin{equation}
  S_{n} = \frac{n}{2} \cdot (a_{1} + a_{n})
\end{equation}

Onde $a_{1}$ é o termo inicial, $a_{n}$ é o termo final e $n$ é o número de termos a serem somados. Logo, podemos traduzir essa formula em um loop for que irá calcular cada elemento do vetor:

\begin{lstlisting}[language=C]
for(i = 0; i < n; i++)
  a[i] = i*(i+1)/2 ;
\end{lstlisting}



%%% Local Variables:
%%% mode: latex
%%% TeX-master: "../../relatorio"
%%% End:

% !TEX encoding = UTF-8 Unicode
% !TEX root = ../../relatorio.tex

%% Responsavel:
\subsection{Questão 5.8}

Consider the loop:
\begin{lstlisting}[language=C]
a[0] = 0;
for(i = 1; i < n; i++)
  a[i] = a[i - 1] + i ;
\end{lstlisting}

There’s clearly a loop-carried dependence, as the value of a[i] can’t be computed without the value of a[i - 1]. Can you see a way to eliminate this dependence and parallelize the loop?

Analisando o resultado do loop, vemos que o vetor vai armazenar a soma de uma sequência de números, uma série aritmética que pode ser descrita como:

\begin{equation}
  S_{n} = \frac{n}{2} \cdot (a_{1} + a_{n})
\end{equation}

Onde $a_{1}$ é o termo inicial, $a_{n}$ é o termo final e $n$ é o número de termos a serem somados. Logo, podemos traduzir essa formula em um loop for que irá calcular cada elemento do vetor:

\begin{lstlisting}[language=C]
for(i = 0; i < n; i++)
  a[i] = i*(i+1)/2 ;
\end{lstlisting}



%%% Local Variables:
%%% mode: latex
%%% TeX-master: "../../relatorio"
%%% End:

% !TEX encoding = UTF-8 Unicode
% !TEX root = ../../relatorio.tex

%% Responsavel:
\subsection{Questão 5.8}

Consider the loop:
\begin{lstlisting}[language=C]
a[0] = 0;
for(i = 1; i < n; i++)
  a[i] = a[i - 1] + i ;
\end{lstlisting}

There’s clearly a loop-carried dependence, as the value of a[i] can’t be computed without the value of a[i - 1]. Can you see a way to eliminate this dependence and parallelize the loop?

Analisando o resultado do loop, vemos que o vetor vai armazenar a soma de uma sequência de números, uma série aritmética que pode ser descrita como:

\begin{equation}
  S_{n} = \frac{n}{2} \cdot (a_{1} + a_{n})
\end{equation}

Onde $a_{1}$ é o termo inicial, $a_{n}$ é o termo final e $n$ é o número de termos a serem somados. Logo, podemos traduzir essa formula em um loop for que irá calcular cada elemento do vetor:

\begin{lstlisting}[language=C]
for(i = 0; i < n; i++)
  a[i] = i*(i+1)/2 ;
\end{lstlisting}



%%% Local Variables:
%%% mode: latex
%%% TeX-master: "../../relatorio"
%%% End:

% !TEX encoding = UTF-8 Unicode
% !TEX root = ../../relatorio.tex

%% Responsavel:
\subsection{Questão 5.8}

Consider the loop:
\begin{lstlisting}[language=C]
a[0] = 0;
for(i = 1; i < n; i++)
  a[i] = a[i - 1] + i ;
\end{lstlisting}

There’s clearly a loop-carried dependence, as the value of a[i] can’t be computed without the value of a[i - 1]. Can you see a way to eliminate this dependence and parallelize the loop?

Analisando o resultado do loop, vemos que o vetor vai armazenar a soma de uma sequência de números, uma série aritmética que pode ser descrita como:

\begin{equation}
  S_{n} = \frac{n}{2} \cdot (a_{1} + a_{n})
\end{equation}

Onde $a_{1}$ é o termo inicial, $a_{n}$ é o termo final e $n$ é o número de termos a serem somados. Logo, podemos traduzir essa formula em um loop for que irá calcular cada elemento do vetor:

\begin{lstlisting}[language=C]
for(i = 0; i < n; i++)
  a[i] = i*(i+1)/2 ;
\end{lstlisting}



%%% Local Variables:
%%% mode: latex
%%% TeX-master: "../../relatorio"
%%% End:

% !TEX encoding = UTF-8 Unicode
% !TEX root = ../../relatorio.tex

%% Responsavel:
\subsection{Questão 5.8}

Consider the loop:
\begin{lstlisting}[language=C]
a[0] = 0;
for(i = 1; i < n; i++)
  a[i] = a[i - 1] + i ;
\end{lstlisting}

There’s clearly a loop-carried dependence, as the value of a[i] can’t be computed without the value of a[i - 1]. Can you see a way to eliminate this dependence and parallelize the loop?

Analisando o resultado do loop, vemos que o vetor vai armazenar a soma de uma sequência de números, uma série aritmética que pode ser descrita como:

\begin{equation}
  S_{n} = \frac{n}{2} \cdot (a_{1} + a_{n})
\end{equation}

Onde $a_{1}$ é o termo inicial, $a_{n}$ é o termo final e $n$ é o número de termos a serem somados. Logo, podemos traduzir essa formula em um loop for que irá calcular cada elemento do vetor:

\begin{lstlisting}[language=C]
for(i = 0; i < n; i++)
  a[i] = i*(i+1)/2 ;
\end{lstlisting}



%%% Local Variables:
%%% mode: latex
%%% TeX-master: "../../relatorio"
%%% End:

% !TEX encoding = UTF-8 Unicode
% !TEX root = ../../relatorio.tex

%% Responsavel:
\subsection{Questão 5.8}

Consider the loop:
\begin{lstlisting}[language=C]
a[0] = 0;
for(i = 1; i < n; i++)
  a[i] = a[i - 1] + i ;
\end{lstlisting}

There’s clearly a loop-carried dependence, as the value of a[i] can’t be computed without the value of a[i - 1]. Can you see a way to eliminate this dependence and parallelize the loop?

Analisando o resultado do loop, vemos que o vetor vai armazenar a soma de uma sequência de números, uma série aritmética que pode ser descrita como:

\begin{equation}
  S_{n} = \frac{n}{2} \cdot (a_{1} + a_{n})
\end{equation}

Onde $a_{1}$ é o termo inicial, $a_{n}$ é o termo final e $n$ é o número de termos a serem somados. Logo, podemos traduzir essa formula em um loop for que irá calcular cada elemento do vetor:

\begin{lstlisting}[language=C]
for(i = 0; i < n; i++)
  a[i] = i*(i+1)/2 ;
\end{lstlisting}



%%% Local Variables:
%%% mode: latex
%%% TeX-master: "../../relatorio"
%%% End:

% !TEX encoding = UTF-8 Unicode
% !TEX root = ../../relatorio.tex

%% Responsavel:
\subsection{Questão 5.8}

Consider the loop:
\begin{lstlisting}[language=C]
a[0] = 0;
for(i = 1; i < n; i++)
  a[i] = a[i - 1] + i ;
\end{lstlisting}

There’s clearly a loop-carried dependence, as the value of a[i] can’t be computed without the value of a[i - 1]. Can you see a way to eliminate this dependence and parallelize the loop?

Analisando o resultado do loop, vemos que o vetor vai armazenar a soma de uma sequência de números, uma série aritmética que pode ser descrita como:

\begin{equation}
  S_{n} = \frac{n}{2} \cdot (a_{1} + a_{n})
\end{equation}

Onde $a_{1}$ é o termo inicial, $a_{n}$ é o termo final e $n$ é o número de termos a serem somados. Logo, podemos traduzir essa formula em um loop for que irá calcular cada elemento do vetor:

\begin{lstlisting}[language=C]
for(i = 0; i < n; i++)
  a[i] = i*(i+1)/2 ;
\end{lstlisting}



%%% Local Variables:
%%% mode: latex
%%% TeX-master: "../../relatorio"
%%% End:

% !TEX encoding = UTF-8 Unicode
% !TEX root = ../../relatorio.tex

%% Responsavel:
\subsection{Questão 5.8}

Consider the loop:
\begin{lstlisting}[language=C]
a[0] = 0;
for(i = 1; i < n; i++)
  a[i] = a[i - 1] + i ;
\end{lstlisting}

There’s clearly a loop-carried dependence, as the value of a[i] can’t be computed without the value of a[i - 1]. Can you see a way to eliminate this dependence and parallelize the loop?

Analisando o resultado do loop, vemos que o vetor vai armazenar a soma de uma sequência de números, uma série aritmética que pode ser descrita como:

\begin{equation}
  S_{n} = \frac{n}{2} \cdot (a_{1} + a_{n})
\end{equation}

Onde $a_{1}$ é o termo inicial, $a_{n}$ é o termo final e $n$ é o número de termos a serem somados. Logo, podemos traduzir essa formula em um loop for que irá calcular cada elemento do vetor:

\begin{lstlisting}[language=C]
for(i = 0; i < n; i++)
  a[i] = i*(i+1)/2 ;
\end{lstlisting}



%%% Local Variables:
%%% mode: latex
%%% TeX-master: "../../relatorio"
%%% End:

% !TEX encoding = UTF-8 Unicode
% !TEX root = ../../relatorio.tex

%% Responsavel:
\subsection{Questão 5.8}

Consider the loop:
\begin{lstlisting}[language=C]
a[0] = 0;
for(i = 1; i < n; i++)
  a[i] = a[i - 1] + i ;
\end{lstlisting}

There’s clearly a loop-carried dependence, as the value of a[i] can’t be computed without the value of a[i - 1]. Can you see a way to eliminate this dependence and parallelize the loop?

Analisando o resultado do loop, vemos que o vetor vai armazenar a soma de uma sequência de números, uma série aritmética que pode ser descrita como:

\begin{equation}
  S_{n} = \frac{n}{2} \cdot (a_{1} + a_{n})
\end{equation}

Onde $a_{1}$ é o termo inicial, $a_{n}$ é o termo final e $n$ é o número de termos a serem somados. Logo, podemos traduzir essa formula em um loop for que irá calcular cada elemento do vetor:

\begin{lstlisting}[language=C]
for(i = 0; i < n; i++)
  a[i] = i*(i+1)/2 ;
\end{lstlisting}



%%% Local Variables:
%%% mode: latex
%%% TeX-master: "../../relatorio"
%%% End:

% !TEX encoding = UTF-8 Unicode
% !TEX root = ../../relatorio.tex

%% Responsavel:
\subsection{Questão 5.8}

Consider the loop:
\begin{lstlisting}[language=C]
a[0] = 0;
for(i = 1; i < n; i++)
  a[i] = a[i - 1] + i ;
\end{lstlisting}

There’s clearly a loop-carried dependence, as the value of a[i] can’t be computed without the value of a[i - 1]. Can you see a way to eliminate this dependence and parallelize the loop?

Analisando o resultado do loop, vemos que o vetor vai armazenar a soma de uma sequência de números, uma série aritmética que pode ser descrita como:

\begin{equation}
  S_{n} = \frac{n}{2} \cdot (a_{1} + a_{n})
\end{equation}

Onde $a_{1}$ é o termo inicial, $a_{n}$ é o termo final e $n$ é o número de termos a serem somados. Logo, podemos traduzir essa formula em um loop for que irá calcular cada elemento do vetor:

\begin{lstlisting}[language=C]
for(i = 0; i < n; i++)
  a[i] = i*(i+1)/2 ;
\end{lstlisting}



%%% Local Variables:
%%% mode: latex
%%% TeX-master: "../../relatorio"
%%% End:

% !TEX encoding = UTF-8 Unicode
% !TEX root = ../../relatorio.tex

%% Responsavel:
\subsection{Questão 5.8}

Consider the loop:
\begin{lstlisting}[language=C]
a[0] = 0;
for(i = 1; i < n; i++)
  a[i] = a[i - 1] + i ;
\end{lstlisting}

There’s clearly a loop-carried dependence, as the value of a[i] can’t be computed without the value of a[i - 1]. Can you see a way to eliminate this dependence and parallelize the loop?

Analisando o resultado do loop, vemos que o vetor vai armazenar a soma de uma sequência de números, uma série aritmética que pode ser descrita como:

\begin{equation}
  S_{n} = \frac{n}{2} \cdot (a_{1} + a_{n})
\end{equation}

Onde $a_{1}$ é o termo inicial, $a_{n}$ é o termo final e $n$ é o número de termos a serem somados. Logo, podemos traduzir essa formula em um loop for que irá calcular cada elemento do vetor:

\begin{lstlisting}[language=C]
for(i = 0; i < n; i++)
  a[i] = i*(i+1)/2 ;
\end{lstlisting}



%%% Local Variables:
%%% mode: latex
%%% TeX-master: "../../relatorio"
%%% End:

% !TEX encoding = UTF-8 Unicode
% !TEX root = ../../relatorio.tex

%% Responsavel:
\subsection{Questão 5.8}

Consider the loop:
\begin{lstlisting}[language=C]
a[0] = 0;
for(i = 1; i < n; i++)
  a[i] = a[i - 1] + i ;
\end{lstlisting}

There’s clearly a loop-carried dependence, as the value of a[i] can’t be computed without the value of a[i - 1]. Can you see a way to eliminate this dependence and parallelize the loop?

Analisando o resultado do loop, vemos que o vetor vai armazenar a soma de uma sequência de números, uma série aritmética que pode ser descrita como:

\begin{equation}
  S_{n} = \frac{n}{2} \cdot (a_{1} + a_{n})
\end{equation}

Onde $a_{1}$ é o termo inicial, $a_{n}$ é o termo final e $n$ é o número de termos a serem somados. Logo, podemos traduzir essa formula em um loop for que irá calcular cada elemento do vetor:

\begin{lstlisting}[language=C]
for(i = 0; i < n; i++)
  a[i] = i*(i+1)/2 ;
\end{lstlisting}



%%% Local Variables:
%%% mode: latex
%%% TeX-master: "../../relatorio"
%%% End:

% !TEX encoding = UTF-8 Unicode
% !TEX root = ../../relatorio.tex

%% Responsavel:
\subsection{Questão 5.8}

Consider the loop:
\begin{lstlisting}[language=C]
a[0] = 0;
for(i = 1; i < n; i++)
  a[i] = a[i - 1] + i ;
\end{lstlisting}

There’s clearly a loop-carried dependence, as the value of a[i] can’t be computed without the value of a[i - 1]. Can you see a way to eliminate this dependence and parallelize the loop?

Analisando o resultado do loop, vemos que o vetor vai armazenar a soma de uma sequência de números, uma série aritmética que pode ser descrita como:

\begin{equation}
  S_{n} = \frac{n}{2} \cdot (a_{1} + a_{n})
\end{equation}

Onde $a_{1}$ é o termo inicial, $a_{n}$ é o termo final e $n$ é o número de termos a serem somados. Logo, podemos traduzir essa formula em um loop for que irá calcular cada elemento do vetor:

\begin{lstlisting}[language=C]
for(i = 0; i < n; i++)
  a[i] = i*(i+1)/2 ;
\end{lstlisting}



%%% Local Variables:
%%% mode: latex
%%% TeX-master: "../../relatorio"
%%% End:

% !TEX encoding = UTF-8 Unicode
% !TEX root = ../../relatorio.tex

%% Responsavel:
\subsection{Questão 5.8}

Consider the loop:
\begin{lstlisting}[language=C]
a[0] = 0;
for(i = 1; i < n; i++)
  a[i] = a[i - 1] + i ;
\end{lstlisting}

There’s clearly a loop-carried dependence, as the value of a[i] can’t be computed without the value of a[i - 1]. Can you see a way to eliminate this dependence and parallelize the loop?

Analisando o resultado do loop, vemos que o vetor vai armazenar a soma de uma sequência de números, uma série aritmética que pode ser descrita como:

\begin{equation}
  S_{n} = \frac{n}{2} \cdot (a_{1} + a_{n})
\end{equation}

Onde $a_{1}$ é o termo inicial, $a_{n}$ é o termo final e $n$ é o número de termos a serem somados. Logo, podemos traduzir essa formula em um loop for que irá calcular cada elemento do vetor:

\begin{lstlisting}[language=C]
for(i = 0; i < n; i++)
  a[i] = i*(i+1)/2 ;
\end{lstlisting}



%%% Local Variables:
%%% mode: latex
%%% TeX-master: "../../relatorio"
%%% End:

% !TEX encoding = UTF-8 Unicode
% !TEX root = ../../relatorio.tex

%% Responsavel:
\subsection{Questão 5.8}

Consider the loop:
\begin{lstlisting}[language=C]
a[0] = 0;
for(i = 1; i < n; i++)
  a[i] = a[i - 1] + i ;
\end{lstlisting}

There’s clearly a loop-carried dependence, as the value of a[i] can’t be computed without the value of a[i - 1]. Can you see a way to eliminate this dependence and parallelize the loop?

Analisando o resultado do loop, vemos que o vetor vai armazenar a soma de uma sequência de números, uma série aritmética que pode ser descrita como:

\begin{equation}
  S_{n} = \frac{n}{2} \cdot (a_{1} + a_{n})
\end{equation}

Onde $a_{1}$ é o termo inicial, $a_{n}$ é o termo final e $n$ é o número de termos a serem somados. Logo, podemos traduzir essa formula em um loop for que irá calcular cada elemento do vetor:

\begin{lstlisting}[language=C]
for(i = 0; i < n; i++)
  a[i] = i*(i+1)/2 ;
\end{lstlisting}



%%% Local Variables:
%%% mode: latex
%%% TeX-master: "../../relatorio"
%%% End:

% !TEX encoding = UTF-8 Unicode
% !TEX root = ../../relatorio.tex

%% Responsavel:
\subsection{Questão 5.8}

Consider the loop:
\begin{lstlisting}[language=C]
a[0] = 0;
for(i = 1; i < n; i++)
  a[i] = a[i - 1] + i ;
\end{lstlisting}

There’s clearly a loop-carried dependence, as the value of a[i] can’t be computed without the value of a[i - 1]. Can you see a way to eliminate this dependence and parallelize the loop?

Analisando o resultado do loop, vemos que o vetor vai armazenar a soma de uma sequência de números, uma série aritmética que pode ser descrita como:

\begin{equation}
  S_{n} = \frac{n}{2} \cdot (a_{1} + a_{n})
\end{equation}

Onde $a_{1}$ é o termo inicial, $a_{n}$ é o termo final e $n$ é o número de termos a serem somados. Logo, podemos traduzir essa formula em um loop for que irá calcular cada elemento do vetor:

\begin{lstlisting}[language=C]
for(i = 0; i < n; i++)
  a[i] = i*(i+1)/2 ;
\end{lstlisting}



%%% Local Variables:
%%% mode: latex
%%% TeX-master: "../../relatorio"
%%% End:

% !TEX encoding = UTF-8 Unicode
% !TEX root = ../../relatorio.tex

%% Responsavel:
\subsection{Questão 5.8}

Consider the loop:
\begin{lstlisting}[language=C]
a[0] = 0;
for(i = 1; i < n; i++)
  a[i] = a[i - 1] + i ;
\end{lstlisting}

There’s clearly a loop-carried dependence, as the value of a[i] can’t be computed without the value of a[i - 1]. Can you see a way to eliminate this dependence and parallelize the loop?

Analisando o resultado do loop, vemos que o vetor vai armazenar a soma de uma sequência de números, uma série aritmética que pode ser descrita como:

\begin{equation}
  S_{n} = \frac{n}{2} \cdot (a_{1} + a_{n})
\end{equation}

Onde $a_{1}$ é o termo inicial, $a_{n}$ é o termo final e $n$ é o número de termos a serem somados. Logo, podemos traduzir essa formula em um loop for que irá calcular cada elemento do vetor:

\begin{lstlisting}[language=C]
for(i = 0; i < n; i++)
  a[i] = i*(i+1)/2 ;
\end{lstlisting}



%%% Local Variables:
%%% mode: latex
%%% TeX-master: "../../relatorio"
%%% End:

% !TEX encoding = UTF-8 Unicode
% !TEX root = ../../relatorio.tex

%% Responsavel:
\subsection{Questão 5.8}

Consider the loop:
\begin{lstlisting}[language=C]
a[0] = 0;
for(i = 1; i < n; i++)
  a[i] = a[i - 1] + i ;
\end{lstlisting}

There’s clearly a loop-carried dependence, as the value of a[i] can’t be computed without the value of a[i - 1]. Can you see a way to eliminate this dependence and parallelize the loop?

Analisando o resultado do loop, vemos que o vetor vai armazenar a soma de uma sequência de números, uma série aritmética que pode ser descrita como:

\begin{equation}
  S_{n} = \frac{n}{2} \cdot (a_{1} + a_{n})
\end{equation}

Onde $a_{1}$ é o termo inicial, $a_{n}$ é o termo final e $n$ é o número de termos a serem somados. Logo, podemos traduzir essa formula em um loop for que irá calcular cada elemento do vetor:

\begin{lstlisting}[language=C]
for(i = 0; i < n; i++)
  a[i] = i*(i+1)/2 ;
\end{lstlisting}



%%% Local Variables:
%%% mode: latex
%%% TeX-master: "../../relatorio"
%%% End:

%% !TEX encoding = UTF-8 Unicode
% !TEX root = ../../relatorio.tex

%% Responsavel:
\subsection{Questão 5.8}

Consider the loop:
\begin{lstlisting}[language=C]
a[0] = 0;
for(i = 1; i < n; i++)
  a[i] = a[i - 1] + i ;
\end{lstlisting}

There’s clearly a loop-carried dependence, as the value of a[i] can’t be computed without the value of a[i - 1]. Can you see a way to eliminate this dependence and parallelize the loop?

Analisando o resultado do loop, vemos que o vetor vai armazenar a soma de uma sequência de números, uma série aritmética que pode ser descrita como:

\begin{equation}
  S_{n} = \frac{n}{2} \cdot (a_{1} + a_{n})
\end{equation}

Onde $a_{1}$ é o termo inicial, $a_{n}$ é o termo final e $n$ é o número de termos a serem somados. Logo, podemos traduzir essa formula em um loop for que irá calcular cada elemento do vetor:

\begin{lstlisting}[language=C]
for(i = 0; i < n; i++)
  a[i] = i*(i+1)/2 ;
\end{lstlisting}



%%% Local Variables:
%%% mode: latex
%%% TeX-master: "../../relatorio"
%%% End:

%% !TEX encoding = UTF-8 Unicode
% !TEX root = ../../relatorio.tex

%% Responsavel:
\subsection{Questão 5.8}

Consider the loop:
\begin{lstlisting}[language=C]
a[0] = 0;
for(i = 1; i < n; i++)
  a[i] = a[i - 1] + i ;
\end{lstlisting}

There’s clearly a loop-carried dependence, as the value of a[i] can’t be computed without the value of a[i - 1]. Can you see a way to eliminate this dependence and parallelize the loop?

Analisando o resultado do loop, vemos que o vetor vai armazenar a soma de uma sequência de números, uma série aritmética que pode ser descrita como:

\begin{equation}
  S_{n} = \frac{n}{2} \cdot (a_{1} + a_{n})
\end{equation}

Onde $a_{1}$ é o termo inicial, $a_{n}$ é o termo final e $n$ é o número de termos a serem somados. Logo, podemos traduzir essa formula em um loop for que irá calcular cada elemento do vetor:

\begin{lstlisting}[language=C]
for(i = 0; i < n; i++)
  a[i] = i*(i+1)/2 ;
\end{lstlisting}



%%% Local Variables:
%%% mode: latex
%%% TeX-master: "../../relatorio"
%%% End:

%% !TEX encoding = UTF-8 Unicode
% !TEX root = ../../relatorio.tex

%% Responsavel:
\subsection{Questão 5.8}

Consider the loop:
\begin{lstlisting}[language=C]
a[0] = 0;
for(i = 1; i < n; i++)
  a[i] = a[i - 1] + i ;
\end{lstlisting}

There’s clearly a loop-carried dependence, as the value of a[i] can’t be computed without the value of a[i - 1]. Can you see a way to eliminate this dependence and parallelize the loop?

Analisando o resultado do loop, vemos que o vetor vai armazenar a soma de uma sequência de números, uma série aritmética que pode ser descrita como:

\begin{equation}
  S_{n} = \frac{n}{2} \cdot (a_{1} + a_{n})
\end{equation}

Onde $a_{1}$ é o termo inicial, $a_{n}$ é o termo final e $n$ é o número de termos a serem somados. Logo, podemos traduzir essa formula em um loop for que irá calcular cada elemento do vetor:

\begin{lstlisting}[language=C]
for(i = 0; i < n; i++)
  a[i] = i*(i+1)/2 ;
\end{lstlisting}



%%% Local Variables:
%%% mode: latex
%%% TeX-master: "../../relatorio"
%%% End:

%% !TEX encoding = UTF-8 Unicode
% !TEX root = ../../relatorio.tex

%% Responsavel:
\subsection{Questão 5.8}

Consider the loop:
\begin{lstlisting}[language=C]
a[0] = 0;
for(i = 1; i < n; i++)
  a[i] = a[i - 1] + i ;
\end{lstlisting}

There’s clearly a loop-carried dependence, as the value of a[i] can’t be computed without the value of a[i - 1]. Can you see a way to eliminate this dependence and parallelize the loop?

Analisando o resultado do loop, vemos que o vetor vai armazenar a soma de uma sequência de números, uma série aritmética que pode ser descrita como:

\begin{equation}
  S_{n} = \frac{n}{2} \cdot (a_{1} + a_{n})
\end{equation}

Onde $a_{1}$ é o termo inicial, $a_{n}$ é o termo final e $n$ é o número de termos a serem somados. Logo, podemos traduzir essa formula em um loop for que irá calcular cada elemento do vetor:

\begin{lstlisting}[language=C]
for(i = 0; i < n; i++)
  a[i] = i*(i+1)/2 ;
\end{lstlisting}



%%% Local Variables:
%%% mode: latex
%%% TeX-master: "../../relatorio"
%%% End:

%% !TEX encoding = UTF-8 Unicode
% !TEX root = ../../relatorio.tex

%% Responsavel:
\subsection{Questão 5.8}

Consider the loop:
\begin{lstlisting}[language=C]
a[0] = 0;
for(i = 1; i < n; i++)
  a[i] = a[i - 1] + i ;
\end{lstlisting}

There’s clearly a loop-carried dependence, as the value of a[i] can’t be computed without the value of a[i - 1]. Can you see a way to eliminate this dependence and parallelize the loop?

Analisando o resultado do loop, vemos que o vetor vai armazenar a soma de uma sequência de números, uma série aritmética que pode ser descrita como:

\begin{equation}
  S_{n} = \frac{n}{2} \cdot (a_{1} + a_{n})
\end{equation}

Onde $a_{1}$ é o termo inicial, $a_{n}$ é o termo final e $n$ é o número de termos a serem somados. Logo, podemos traduzir essa formula em um loop for que irá calcular cada elemento do vetor:

\begin{lstlisting}[language=C]
for(i = 0; i < n; i++)
  a[i] = i*(i+1)/2 ;
\end{lstlisting}



%%% Local Variables:
%%% mode: latex
%%% TeX-master: "../../relatorio"
%%% End:



\newpage

\bibliographystyle{abbrv}
\bibliography{
	sections/modelo/ref.bib
	}
\end{document}
